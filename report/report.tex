\documentclass{article}
\usepackage{pdfpages}
\usepackage[utf8]{inputenc}
\usepackage[english]{babel}
\usepackage{hyperref}
\usepackage{apacite}
\usepackage{mathptmx}
\usepackage[font = {small, it}]{caption}
\DeclareCaptionLabelFormat{cont}{#1~#2\alph{ContinuedFloat}}
\captionsetup[ContinuedFloat]{labelformat=cont}
\usepackage{subcaption}
\usepackage{float}
\usepackage{fancyhdr}
\usepackage{graphicx}
\usepackage{amsmath}
\setlength{\parindent}{2em}
\setlength{\parskip}{1em}
\renewcommand{\baselinestretch}{1.5}


\fancypagestyle{plain}{
\fancyhf{}
\rhead{Mikkel Werling (201706722) \\ Sebastian Scott Engen (201708490)}
\lhead{The Prospects of Emotion}
\cfoot{\thepage}
}
\pagestyle{plain}

\title{The Prospects of Emotion \\
\large Investigating the modulating effects of emotions on the parameters of Cumulative Prospect Theory
}
\author{Mikkel Werling (201706722) \\ Sebastian Scott Engen (201708490)}
\date{January 2021}

\begin{document}
    \maketitle
    \tableofcontents
    \section{Introduction}
    \subsection{Prospect Theory}
    \subsubsection{Prospect Theory and Expected Utility}
    \subsubsection{Cumulative Prospect Theory (CPT)}
    \subsubsection{Replication}
    \subsubsection{The parameters of Cumulative Prospect Theory}
    \subsection{Emotions}
    \subsubsection{Rationality and Emotions}
    \subsubsection{Historical shift in understanding emotions}
    \subsubsection{Eliciting Emotions}
    \subsubsection{Videos}
    \subsection{Prospect Theory under the influence of Emotion}
    \subsection{Models and Procedure}
    \subsubsection{Estimating parameters}
    \subsubsection{Bayesian Approach and Results}
    \subsection{Hypotheses}
    \subsubsection{Manipulation checks}
    \subsubsection{Cross study check}
    \subsubsection{Hypothesis testing}

    \section{Design Plan}
    \subsection{Blinding}
    \subsection{Study Design}
    \subsection{Randomization}
    \section{Sampling Plan}
    \subsection{Data Collection Procedures}
    \subsection{Sample Size}
    \subsection{Stopping Rule}
    \section{Variables}
    \subsection{Manipulated Variables}
    \subsection{Measured Variables}
    \subsection{Indices}
    \section{Analysis Plan}
    \subsection{Statistical models}
    \subsection{Transformations}
    \subsection{Inference criteria}
    \subsection{Data Exclusion}





    $$V(O) = \sum \pi (p_i)v(x_i) $$

    $$\pi(p_i) = \frac{p_i^c}{(p_i^c-[1-p_i^c])^\frac{1}{c}}$$

    $$v(x) = \begin{cases}
        x^\alpha & \text{if } x \geq 0 \\
        -\lambda(-x)^\beta & \text{if } x < 0 
    \end{cases}$$
\end{document}